
micro cortical model; numbers

Assume 80,000 neurons, each with 5,000 synapses, firing at 10Hz. This
gives 400,000,000 synapses.  If each synapse requires order 4 bytes to
store, then this corresponds to 1.6 Gbytes, and is suggestive of the
need for at least 12.8 chips (each SpiNNaker-1 chip has a 1Gbit SDRAM,
so we need eight chips per Gbyte.) The MCM therefore comfortably fits onto a
single board.

Then each second we expect 800,000 spikes, and 800,000 x 5,000 =
4,000,000,000 synaptic events.

With a "clock tick" of 0.1ms, we therefore expect 80 spikes, and
400,000 synaptic events over the entire MCM.

Assuming each spike goes everywhere, then each core has to handle 80
spikes.

If we are using a 48 node board then each CPU has to cope with an
average of 521 synaptic events per clock tick.  The memory for the
synapses is 520,833 synapses per core, or 2 Mbytes (at 4 bytes per
synapse). Also we need 104 neurons per board. This is all eminently
achievable.

\section{A 12 chip MCM?}

How about if we try to do this with just 12 chips?

Then, we get 416 neurons per core; 2,084 synaptic events per clock
tick, and memory for the synapses runs to 2.0833 Mbytes -- which is
fractionally above the 2Mbyte per core limit.


DMA limits: 640 Mbyte/s for SDRAM, this means that each core can
access on average 40Mbyte/s or 4k byte per clock tick. This is just
1000 synapses per clock tick -- unless we introduce a fancy new
synaptic row representation.


Each Excitatory delay is in the range: 0.8 - 2.2 ms (delays 8,9,10.. 22)
Each Inhibitory delay is in the range: 0.4 - 1.2 ms (delays 4,5, .. 12)

If a spike rowlet covers just four delays, then associated with each Ex spike we have 6 rowlets;
and each Inh spike has 3 (but almost 2) rowlets.

Total rowlets in play at any one time is: 3x2x8 = 48 Inh, and 6x4x8 =
192 Ex; total = 240.


Generating some rowlets in python

>>> def syn_ex (p):
...     ns = [n for x, n in sorted (zip (np.random.uniform (0,1,512), range (512))) if x < p]
...     ws = [int(round(w*10)) for w in np.random.normal (768, 7.68, len(ns))]
...     ds = sorted ([int(round(d*10)) for d in np.random.normal (1.5, 0.75, len(ns))])
...     return (ns, ws, ds)     


                L2/3e	L2/3i	L4e		L4i    	L5e   	L5i            L6e	        L6i	       Th
                20683 	5834 	21915 	5479 	4850 	1065 	14395      2948 	902 = 77169+902
                80.79       22.78      85.6          21.4         18.9         4.16          56.23       11.51

                                                                  from				
                L2/3e	L2/3i	L4e		L4i    	L5e   	L5i            L6e	        L6i	       Th
to  L2/3e	0.101	0.169	0.044	0.082	0.032	0.0	        0.008	0.0   	0.0
     L2/3i	0.135	0.137	0.032	0.052	0.075	0.0	        0.004	0.0    	0.0
     L4e	0.008	0.006	0.050	0.135	0.007	0.0003	0.045	0.0   	0.0983
     L4i	0.069	0.003	0.079	0.160	0.003	0.0	        0.106	0.0   	0.0619
     L5e	0.100	0.062	0.051	0.006	0.083	0.373	0.020	0.0	        0.0
     L5i	0.055	0.027	0.026	0.002	0.060	0.316	0.009	0.0    	0.0
     L6e	0.016	0.007	0.021	0.017	0.057	0.020	0.040	0.225	0.0512
     L6i	0.036	0.001	0.003	0.001	0.028	0.008	0.066	0.144	0.0196

Name	Value	Description
w ± δw	87.8 ± 8.8 pA	Excitatory synaptic strengths sd = 0.1 apart
                                        from  L4e and L2/3e which is 0.05

g	–4	Relative inhibitory synaptic strength
de ± δde	1.5 ± 0.75 ms	Excitatory synaptic transmission delays
di ± δdi	0.8 ± 0.4 ms	Inhibitory synaptic transmission delays

Neuron model
Name	Value	Description
τm	10 ms	Membrane time constant
τref	2 ms	Absolute refractory period
τsyn	0.5 ms	Postsynaptic current time constant
Cm	250 pF	Membrane capacity
V_reset	−65 mV	Reset potential
θ	−50 mV	Fixed firing threshold
νth	15 Hz	Thalamic firing rate during input period
